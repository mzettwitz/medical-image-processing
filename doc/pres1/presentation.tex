\documentclass[11pt]{beamer}
\usetheme{Rochester}
\usecolortheme{seagull}
\usepackage[utf8]{inputenc}
\usepackage[german]{babel}
\usepackage[T1]{fontenc}
\usepackage{amsmath}
\usepackage{amsfonts}
\usepackage{amssymb}
\usepackage{textpos}
\usepackage{subcaption}
\usepackage{wrapfig}

% add logo to the sections
\addtobeamertemplate{frametitle}{}{%
\begin{textblock*}{100mm}(.6\textwidth,-1.2cm)
\includegraphics[width=0.5\linewidth]{./logo_inf_fak.png}
\end{textblock*}}


\setbeamertemplate{footline}[frame number]

\author{Zettwitz, Grabe, Niemann}
\title{Automatisierte Detektion einer Biopsienadel \\ in CT-Bildatensätzen}
\date{10.Mai 2016} 
%\setbeamercovered{transparent}  
%\subject{}


\begin{document}

\begin{frame}
\titlepage
\end{frame}


\section{Situation}
\begin{frame}
\frametitle{Situation}
\begin{itemize} 
\item DICOM Datensatz mit CT Daten
\item Biopsienadel in Schichtbildern
\end{itemize}

\begin{figure}[H]
\includegraphics[width=\textwidth]{../../../data/presentation/pre1/original_p1.png}
\end{figure} 
\end{frame}


\subsection{Probleme}
\begin{frame}
\frametitle{Probleme}
\begin{itemize}
\item Nadel strahlt
\item Rauschen in Bildern
\item Nadel ist teilweise nicht in Schicht sichtbar
\item Nadelspitze unterbrochen
\item Knochen und Nadel haben ähnliche Hounsfield Werte
\end{itemize}

\begin{figure}[H]
\includegraphics[width=0.85\textwidth]{../../../data/presentation/pre1/original_spray_p1.png}
\end{figure} 
\end{frame}

\section{Konzept}
\begin{frame}
\frametitle{Lösungsansätze}
\begin{itemize}
\item Window/Leveling
\item Morphologisches closing
\item Anisotroper Diffusionsfilter
\item Hough Transformation
\item Markieren der längsten Strecke/Ausgleichsgerade
\end{itemize}
\end{frame}


\section{Erste Ergebnisse}
\subsection{Bisherige Umsetzunng}
\begin{frame}
\frametitle{Bisherige Umsetzung}
\begin{itemize}
\item Einlesen aller Bilder aus einem Datensatz
\item Filterung(Anisotroph, Statistisch)
\item Hough Transformation mit Ausgabe: Gerade, Spitze
\end{itemize}
\end{frame}

\subsection{Erste Ergebnisse}
\begin{frame}[allowframebreaks]
\frametitle{Erste Ergebnisse}
\begin{itemize}
\item Preprocessing bringt signifikanten Vorteile bei der HT
\item Trennen von Gewebe und Nadel meist zuverlässig(P01)
\item Erkennen der Nadel zuverlässig wenn tiefer im Gewebe
\item Knochen werden teilweise als Nadel interpretiert
\end{itemize}


\framebreak
\begin{figure}[H]
\centering
\includegraphics[height=0.85\textheight]
{../../../data/presentation/pre1/hough_simple_p1.png}
\caption{Hough Transformation}
\end{figure}

\framebreak
\begin{figure}[H]
\centering
\includegraphics[height=0.85\textheight]
{../../../data/presentation/pre1/hough_simple_fail_p1.png}
\caption{Hough Transformation Fehler}
\end{figure}

\end{frame}

%BIld (HT ohne prepro)
%mit prepro
%Ausgleichsgerade
%gutes + schlechtes Bild


\end{document}
